\documentclass{article}
\usepackage{amsmath,amssymb,amsthm} % AMS styles for extra equation formatting
\usepackage{graphicx} % for including graphics files
\usepackage{subfig} % for subfigures
\usepackage[numbers,sort]{natbib} % for better references control
\usepackage{hyperref} % for hyperlinks within the paper and references
\usepackage{fontspec}  % Allows for system fonts
\usepackage[top=2cm, bottom=2cm, left=2cm, right=2cm]{geometry}  % Set margins on all sides
\usepackage{setspace} % for line spacing
\usepackage{appendix} % for the appendices
\usepackage{listings} % for code
\usepackage{xcolor} % for color
\usepackage{url,textcomp}
\usepackage{matlab-prettifier}
\usepackage{tabularx}
%%%%%%%%%%%%%%%%%%%%%%%%%%%%%%%%%%%%%%%%%%%%%%%%%%%%%%%%%%%%%%%%%%%%%%%%%%%%%%

\hypersetup{colorlinks=true, linkcolor=blue,  anchorcolor=blue,
citecolor=blue, filecolor=blue, menucolor=blue, pagecolor=blue,
urlcolor=blue}

%%%%%%%%%%%%%%%%%%%%%%%%%%%%%%%%%%%%%%%%%%%%%%%%%%%%%%%%%%%%%%%%%%%%%%%%%%%%%%

\newcommand{\todo}[1]{\vspace{5 mm}\par \noindent
\marginpar{\textsc{Todo}}
\framebox{\begin{minipage}[c]{0.90 \textwidth}
\tt \flushleft #1 \end{minipage}}\vspace{5 mm}\par}
\newcommand{\setParDis}{\setlength {\parskip} {0.2cm} } % for 0.3cm spacing
\newcommand{\setParDef}{\setlength {\parskip} {0pt} } % for 0 spacing

%%%%%%%%%%%%%%%%%%%%%%%%%%%%%%%%%%%%%%%%%%%%%%%%%%%%%%%%%%%%%%%%%%%%%%%%%%%%%%

\graphicspath{{graphics/}}

\newtheorem{theorem}{Theorem}[section]
\newtheorem{proposition}[theorem]{Proposition}
\newtheorem{lemma}[theorem]{Lemma}
\newtheorem{corollary}[theorem]{Corollary}
\newtheorem{definition}[theorem]{Definition}

%\renewcommand{\qedsymbol}{$\blacksquare$} % for filled square at end of proof
%\numberwithin{equation}{section} % for the 1.1, 1.2 equation number style
%\setlength{\parindent}{0em} % don't indent paragraphs
%\setlength{\parskip}{1em} % add spacing between paragraphs
%\linespread{1.6} % double-spacing

\setmainfont{Arial}
% \doublespacing
\onehalfspacing
\setcounter{secnumdepth}{3}

%%%%%%%%%%%%%%%%%%%%%%%%%%%%%%%%%%%%%%%%%%%%%%%%%%%%%%%%%%%%%%%%%%%%%%%%%%%%%%

\begin{document}

\title{%
Report for Speech Recognition \\
\large\itshape{EEEM030 - Speech Recognition - Group Assignment 2}}
\author{\normalsize\slshape{Xiaoguang Liang}}
\date{\normalsize\slshape\today}
\maketitle


% Suppress any floats (figures, tables) from appearing on the next page
\suppressfloats

\tableofcontents

\begin{abstract}


\end{abstract}

%%%%%%%%%%%%%%%%%%%%%%%%%%%%%%%%%%%%%%%%%%%%%%%%%%%%%%%%%%%%%%%%%%%%%%%%%%%%%%

\section{Introduction}
\setParDis

This report is structured as follows: \textit{Section 2} analyses the kinematics of the Trossen PincherX-100 Robot. \textit{Section 3} compares the differences between Trossen PincherX-100 Robot and TiRobot II. \textit{Section 4} explores the applications of TiRobot II in detail. Finally, conclusions are drawn in \textit{Section 5}.

%%%%%%%%%%%%%%%%%%%%%%%%%%%%%%%%%%%%%%%%%%%%%%%%%%%%%%%%%%%%%%%%%%%%%%%%%%%%%%

\section{MFCC acoustic features}


\subsection{Model initialization}



\subsection{Model initialization}
Model initialization is a foundational step in training Hidden Markov Models (HMMs). The success of training algorithms like the Baum-Welch algorithm relies heavily on well-initialized model parameters. These parameters include the initial state probabilities $Pi$, the state transition probabilities $A$, and the observation probabilities $B$. Proper initialization provides a good starting point for optimization and helps avoid convergence to suboptimal solutions. 

\subsection{Initializaton for Multivariate Gaussian pdf}

\subsubsection{Segments splitting for data}

\subsubsection{Calculation for mean}

\subsubsection{Calculation for variance}


\section{Model training with HMMs}


\section{Evaluation}


\section{Conclusion}



%%%%%%%%%%%%%%%%%%%%%%%%%%%%%%%%%%%%%%%%%%%%%%%%%%%%%%%%%%%%%%%%%%%%%%%%%%%%%%

\newcommand{\doi}[1]{DOI: \href{http://dx.doi.org/#1}{\nolinkurl{#1}}}
\bibliographystyle{ieeetr}
\bibliography{refs}

%%%%%%%%%%%%%%%%%%%%%%%%%%%%%%%%%%%%%%%%%%%%%%%%%%%%%%%%%%%%%%%%%%%%%%%%%%%%%%

\end{document}
